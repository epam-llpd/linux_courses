\input{report}

\newcommand{\defaultuser}{cur\_user}

\title{Основы Linux}

\begin{document}
\maketitle

\tableofcontents

\chapter{Основы Linux}

\begin{frame}{Основы ОС Linux}

	\begin{block}{Вопрос}
	Почему Linux является самой популярной
	свободной операционной системой?
	\end{block}

	\pause

	\begin{block}{Ответ}
	\begin{itemize}
		\item \textcopyleft -- Copyleft
		\item ``Философия'' Unix
		\item Открытые стандарты
	\end{itemize}
	\end{block}

\end{frame}


%%%%%%%%%%%%%%%%%%%%%%%%%%%%%%%%%%%%%%%%%   
%%%%%%%%%% Content starts here %%%%%%%%%%
%%%%%%%%%%%%%%%%%%%%%%%%%%%%%%%%%%%%%%%%%

\section[Принципы]{Базовые принципы ОС Linux}

\subsection{GNU/Linux}

\mode<all>{\input{../../slides/intro/vocabulary}}

\subsection{Лицензии}

\mode<all>{\input{../../slides/intro/licenses}}

\subsection{Принципы проектирования переносимых программ}

\mode<all>{\input{../../slides/intro/unixway}}

\section{Дистрибутивы ОС Linux}

\mode<all>{\input{../../slides/intro/linux-distro}}

\section{Процесс загрузки ОС Linux}

\subsection{Этапы загрузки}

\mode<all>{\input{../../slides/intro/linux-boot}}

\subsection{Ядро Linux}

\mode<all>{\input{../../slides/intro/linux-kernel}}

\subsection{Userspace}

\mode<all>{\input{../../slides/intro/initrd}}

\mode<all>{\input{../../slides/intro/init-process}}

\subsection{Практика}

\mode<all>{\input{../../slides/intro/practice01.tex}}




\chapter{Командная строка}

\section{Командная оболочка (shell)}

\mode<all>{\begin{frame}[fragile]{Оболочка операционной системы}

     \begin{block}{Оболочка операционной системы}
     (от англ. shell «оболочка») \alert{интерпретатор команд} операционной системы, обеспечивающий интерфейс для взаимодействия пользователя с функциями системы.  
     \end{block}
     Что такое Unix shell?
     \begin{itemize}
        \item Обычная программа, запускающаяся после входа в систему
        \item Интерактивный командный интерпретатор
        \item Платформа интеграции (для утилит)
        \item Язык программирования
     \end{itemize}

	\pause
	\vspace{0.5in}
	Пример shell из  Windows-world --  cmd.exe, PowerShell
	\vspace{0.5in}
	Минимальный дистрибутив Linux -- ядро + shell 

\end{frame}
\note{   
     Как вы помните задачи ядра - управление процессами и потоками, управление
     памятью, Управление файлами, Разграничение доступа, мониторинг и
     конфигурация. Shell обеспечивает пользователю  интерфейс доступа к этим функциям ядра.    
     Как Интерактивный командный интерпретатор shell предосталяет средства:
     редактора командной строки, часто со средствами упрощающими ввод команд
     такими как  горячие клавишами, автодополнение, история команд и поиск по
     истории, сокращения команд (aliases), сокращения путей к файлам c помощью
     спецсимволов. 
     Как Платформа интеграции предоставляет средства пренаправления
     ввода/вывода, каналы (pipes), и возможность создавать скрипты.
     Одновременно является языком программирования, но эту часть будем
     рассматривать позже. 
}

\begin{frame}[fragile]{Задание. Виды оболочек.}
Получить список установленных оболочек.
\begin{lstlisting}[language=bash]
cat /etc/shells
ls -l <filename> # для каждого элемента /etc/shells
readlink -e <filename> 
\end{lstlisting}
Запустить любую из установленных оболочек. 

\begin{lstlisting}[language=bash]
/usr/bin/ksh 
/usr/bin/zsh 
/usr/bin/fish
\end{lstlisting}

выход из оболoчки  Ctrl-D, либо exit
\end{frame}


\begin{frame}[fragile]{Основные типы shell в Unix}
  \begin{itemize}
    \item Bourne shell совместимые (POSIX-совместимые)
      \begin{itemize}
        \item \textbf{sh} исходная bourne shell (Steve Bourne, 1978)
        \item \textbf{ksh} Korn shell (David Korn, 1983)
        \item \textbf{ash} $[$BSD$]$ Almquist shell (Kenneth Almquist,1989)  
        \item \textbf{bash} $[$GPL$]$ Bourne-again shell (Brian Fox, 1989)
        \item \textbf{zsh} $[$BSD$]$ Z shell (Paul Falstad,1990)
        \item \textbf{/bin/sh} Указывает на POSIX-совместимую shell
      \end{itemize}
  \item C shell совместимые
      \begin{itemize}
        \item \textbf{csh}  Исходная С shell (Bill Joy, 1978)
        \item \textbf{tcsh} $[$BSD$]$ TENEX C shell (Ken Greer, 1981)
       \end{itemize}
  \end{itemize}
\end{frame}

\note { 
Существуют  Рассмотрим наиболее распрастраненные реализации оболочек: 
Разделяются на две группы POSIX-совместимые и несовместимые. 
sh — оригинальный шелл Борна;
bash (bourne again shell) (эмуляция совместимости POSIX[1]) расширенная Борном свободная (разработанная в рамках проекта GNU). Стандартная оболочка в Linux.

C shell — (несовместима с POSIX shell) оболочка, с синтаксисом на основе Си,
созданная Университетом Беркли в рамках проекта по реализации BSD Unix.
    csh (C-Shell) — оболочка из состава дистрибутива BSD, имеет Си-образный синтаксис и не является POSIX-совместимой. Впервые введены возможности управления заданиями и произведены другие улучшения.
    tcsh (csh) — реализация csh с интерактивными возможностями, не уступающими bash[1]. Удобна для интерактивной работы. Совместима с csh.
}

\begin{frame}[fragile]{Различия между оболочками}
  \begin{itemize}
    \item Интерактивные возможности,  автодополнение, подсветка синтаксиса, клавиатурные сокращения
    \item Встроенные команды 
    \item Совместимость c POSIX или между собой
    \item Поиск соответствий строк и имён файлов, подстановки, globbing
    \item Средства программирования
\end{itemize}
\end{frame}
/note {
Интерактивные возможности - клавиатурные сокращения, средства автодополнения, настройки по умолчанию
Платформы - на чем работают оболочки. bash - кросплотформенная, zsh - нет. 
Программные возможности - различный синтаксис, bashизмы, функции определяются
по разному, средства программирования циклы, списки, массивы. Отсюда
совместимость между собой. 
Мы будет придерживаться синтаксиса оболочки bash, т.к. она по умолчанию
используется в дистрибутивах Linux.  Это позволит создать базу, с
котороый вы легко сможете перейти на подходящие вам варианты. 
}

\begin{frame}[fragile]{Задание. Виды команд в оболочке}
Выполним команду type для разных команд.
\begin{lstlisting}[language=bash]
type type cd help alias read
type dmesg rm
type if
type -a ls
type -a echo pwd test
\end{lstlisting}
\end{frame}

\begin{frame}[fragile]{Запуск программы из командной строки}
  \begin{itemize}
    \item 
	Находим приглашение командной строки
	\$, \#, user@host:~\$
    \item
	Вводим имя команды, аргументы и запускаем на выполнение нажатием <Enter>
   \end{itemize}

	Что такое команды?
  \begin{itemize}
    \item исполняемая программа (бинарный файл, скрипт)
    \item встроенные в оболочку команды (shell built-ins)
    \item функция оболочки
    \item сокращение команды (an alias) 
  \end{itemize}
\end{frame}
\note {
Система ожидает ввода команды, показывая приглашение командной строки.
Пользователь вводит команды опции аргументы и нажимает <Enter>. 
Опции - модификаторы поведения программы.  
Аргументы - то над чем производятся действия.
}
}

\section{Don't panic! Получение помощи}

\mode<all>{\input{../../slides/cmdline/help}}

\section{Навигация по файловой системе}

\mode<all>{\input{../../slides/cmdline/fs-navigation-cmd}}

\mode<all>{\input{../../slides/fs/fs-structure}}

\section{Полезные инструменты bash}
\mode<all>{\begin{frame}{Важные аббревиатуры внутри командной строки}
  \begin{itemize}
    \item Для директорий
      \begin{itemize}
        \item {\tt $\sim$} Домашняя директория
        \item {\tt $\sim$<username>} Домашняя директория пользователя
        \item {\tt ..} Родительская директория
        \item {\tt .} Текущая директория
      \end{itemize}
      \pause  
    \item Wildcards
      \begin{itemize}
        \item {\tt *} Любой набор символов {\tt file*txt : file1.txt filefilefiletxt}
        \item {\tt $[$<список>$]$ } символ из заданного набора
        \item {\tt ?} любой один символ
      \end{itemize}

  \end{itemize}
\end{frame}


\begin{frame}{Переменные окружения}
  \begin{itemize}
    \item {\tt HOME}
    \item {\tt PWD}
    \item {\tt LANG}
    \item {\tt LD\_LIBRARY\_PATH}
    \item {\tt SHELL}
    \item {\tt TERM}
    \item {\tt DISPLAY}
  \end{itemize}

  Контроль

  \begin{itemize}
    \item {\tt export VAR=value}
    \item declare -x
    \item echo \$VAR получить содержимое переменной
  \end{itemize}

  Переменные окружения наследуются при создании нового процесса
\end{frame}



}

\chapter{Управление процессами и перенаправления}

\section{Процессы}
\mode<all>{\input{../../slides/cmdline/process-intro}}

\section{Перенаправление ввода-вывода}
\mode<all>{\input{../../slides/cmdline/pipes.tex}}
\mode<all>{\input{../../slides/cmdline/io-redirection}}
\mode<all>{\input{../../slides/cmdline/io-redirection-here.tex}}

\chapter{Полезные команды}

\section{Полезные команды}

\mode<all>{\input{../../slides/cmdline/commands1.tex}}

\subsection{Архиваторы}
\mode<all>{\input{../../slides/cmdline/commands2}}

\subsection{find и xargs}
\mode<all>{\input{../../slides/cmdline/commands3}}

\subsection{Редакторы}
\mode<all>{\input{../../slides/cmdline/editors-intro}}

\chapter{Система управления пакетами}
\section{Система управления пакетами}
\mode<all>{\begin{frame}{Сетевая подсистема Linux}

	\begin{block}{Cетевой интерфейс}

		Сетевой интерфейс в Linux -– это абстрактный именованный объект,  используемый для передачи 
		данных через некоторую линию связи без привязки к ее (линии связи) реализации.
	\end{block}
\end{frame}

\begin{frame}{Сетевая подсистема Linux}

	\center\includegraphics[width=0.9\textwidth]{../../slides/networking/06-netstack.png}

\end{frame}


}

\mode<all>{\input{../../slides/packaging/intro1}}

\mode<all>{\input{../../slides/packaging/rpm}}

\mode<all>{\newcounter{tmpc}

\begin{frame}{Репозиторий}
	\begin{block}{Репозиторий пакетов}
		Место, где хранятся и поддерживаются пакеты, а также сопутствующая мета-информация, предназначенное для использования пакетным менеджером.
	\end{block}
	\begin{block}{Пример: Fedora Core}
		\begin{itemize}
			\item Packages/*.rpm
			\item RPM-GPG-KEY-*
			\item repodata
			\begin{itemize}
				\item множество сжатых и несжатых XML файлов для YUM
			\end{itemize}
		\end{itemize}

		Описание репозтория для YUM на локальной системе хранится по пути
		{\tt /etc/yum.repos.d/*.repo}
	\end{block}
		
\end{frame}

\begin{frame}{Apt: команды}
	\begin{block}{Установка/обновление пакета}
		{\tt apt-get install pkgname }

                {\tt apt-get -f install}
	\end{block}
	\begin{block}{Обновление данных о пакетах}
		{\tt apt-get update }
	\end{block}
	\begin{block}{Удаление пакета}
		{\tt apt-get remove pkgname }
	\end{block}
	\begin{block}{Поиск}
		{\tt apt-cache search pkgname }
	\end{block}
\end{frame}

\begin{frame}{YUM: команды}
	\begin{block}{Установка/обновление пакета}
		{\tt yum install pkgname }
	\end{block}
	\begin{block}{Обновление всех пакетов}
		{\tt yum update }
	\end{block}
	\begin{block}{Удаление пакета}
		{\tt yum remove pkgname }
	\end{block}
	\begin{block}{Поиск}
		{\tt yum list pkgname }\\
		{\tt yum search pkgname }
	\end{block}
\end{frame}


\begin{frame}[fragile]{Упражнение}
%  \begin{enumerate}
%      \item Создать на {\tt /dev/sda} раздел размером примерно 10Gb
%      \item Создать на этом разделе ext3 ФС и смонтировать раздел в {\tt /mnt/chroot}
%      \item Развернуть {\tt /media/nfs/pub/CentOS/precreated/centOS.tar.gz} в {\tt /mnt/chroot}
%      \item Смонтировать {\tt proc, sysfs} а также {\tt /dev} в соответствующие места {\tt /mnt/chroot}
%      \item {\tt chroot /mnt/chroot}
%      \item Отредактировать {\tt /etc/resolv.conf} -- скопировать туда информацию из {\tt resolv.conf} основной системы
%      \item Отредактировать {\tt /etc/yum.conf} Добавить следующий раздел
%\begin{minipage}{0.5\textwidth}
%\begin{verbatim}
%[base]
%  name = CentOS 6
%  baseurl = ftp://192.168.11.15/CentOS
%  gpgcheck = 0
%\end{verbatim}
%\end{minipage}
%\setcounter{tmpc}{\theenumi}
%\end{enumerate}
%\end{frame}
%\begin{frame}{Продолжение упражнения}
  \begin{enumerate}
      %\setcounter{enumi}{\thetmpc}
      \item {\tt apt-get update}
      \item Удалить пакет vim
      \item Установить заново пакет vim
      \item Посмотреть списки файлов для пакетов {\tt rpm, vim}
      \item Найти, к какому пакету относится команда {\tt ls, top}
      \item Найти пакет предоставляющий сервис ssh и установить его
    \end{enumerate}
\end{frame}


}

\chapter{Пользователи и привилегии}

\section{Многопользовательская модель UNIX}
\mode<all>{\input{../../slides/multiuser/multiuser-model.tex}}

\mode<all>{\input{../../slides/multiuser/fs-permissions.tex}}
\section{Внутренний механизм управления пользователями}
\mode<all>{\input{../../slides/multiuser/account_files.tex}}

\mode<all>{\input{../../slides/multiuser/pam.tex}}

\chapter{Сетевая подсистема}

\section{Основы работы с сетевой подсистемой}

\mode<all>{\begin{frame}{Сетевая подсистема Linux}

	\begin{block}{Cетевой интерфейс}

		Сетевой интерфейс в Linux -– это абстрактный именованный объект,  используемый для передачи 
		данных через некоторую линию связи без привязки к ее (линии связи) реализации.
	\end{block}
\end{frame}

\begin{frame}{Сетевая подсистема Linux}

	\center\includegraphics[width=0.9\textwidth]{../../slides/networking/06-netstack.png}

\end{frame}


}

\subsection{Управление интерфейсами}
\mode<all>{\input{../../slides/networking/interface-management}}

\subsection{Полезные программы}
\mode<all>{\input{../../slides/networking/useful-progs}}

\section{ssh}
\mode<all>{\input{../../slides/networking/ssh}}


\section{Дополнительные типы интерфейсов}

\subsection{alias, vlan}
\mode<all>{\input{../../slides/networking/alias_vlan}}
\subsection{Мосты}
\mode<all>{\input{../../slides/networking/bridge}}
\subsection{Тоннели}
\mode<all>{\input{../../slides/networking/tuntap}}

\section{Маршрутизация}
\mode<all>{\input{../../slides/networking/routing}}

\section{iptables}
\mode<all>{\begin{frame}{Iptables}

	\center\includegraphics[width=0.9\textwidth]{../../slides/networking/06-iptables.png}

\end{frame}

\begin{frame}{Iptables}

	\center{\bf iptables -t <table> -L}
	\center{\bf iptables -t <table> -F}
	\bigskip

	\begin{itemize}
	\begin{columns}
		\column{0.3\textwidth}

			\item filter -- файерволл
				\begin{itemize}
					\item INPUT
					\item FORWARD
					\item OUTPUT
				\end{itemize}
		\column{0.3\textwidth}
			\item nat -- преобразования адресов
				\begin{itemize}
					\item PREROUTING
					\item INPUT
					\item OUTPUT
					\item POSTROUTING
				\end{itemize}
		\column{0.3\textwidth}
			\item mangle -- специальные  изменения  пакетов (TOS, TTL, MARK)
				\begin{itemize}
					\item PREROUTING
					\item INPUT
					\item FORWARD
					\item OUTPUT
					\item POSTROUTING
				\end{itemize}
		\end{columns}
	\end{itemize}

\end{frame}


\begin{frame}{Iptables: примеры}

	\center{\bf iptables -t <table> <CRITERIA> <TARGET>}
	\small
	\begin{itemize}
		\item filter:\\
			{\tt iptables -A INPUT -s 192.168.0.1/24 -p UDP -j REJECT -{}-reject-with icmp-host-unreachable}\\
			{\tt iptables -A INPUT -d 192.168.0.1/24 -p TCP -j DROP}
		\item nat:\\
			{\tt iptables -A POSTROUTING -t nat -s 192.168.1.0/24 -j MASQUERADE}\\
			{\tt iptables -t nat -A PREROUTING -p tcp -d 192.168.251.1 
			--dport 8080 -{}-sport 1024:65535 -j DNAT -{}-to 192.168.1.200:8080}
		\item mangle:\\
			{\tt iptables -A PREROUTING -t mangle -p tcp -{}-dport 22 -j MARK -{}-set-mark 100}\\
			{\tt ip route add default dev eth0 table 1}\\
			{\tt ip rule add fwmark 100 table 1}
	\end{itemize}

\end{frame}


}

\chapter{Дисковая подсистема}

\section{Блочные устройства}
\mode<all>{\input{../../slides/disk/disk-intro.tex}}
\section{Основные команды}
\mode<all>{\input{../../slides/disk/disk-simple_management.tex}}
\section{GPT}
\mode<all>{\input{../../slides/disk/disk-gpt.tex}}
\section{LVM}
\mode<all>{\input{../../slides/disk/lvm.tex}}


\end{document}
