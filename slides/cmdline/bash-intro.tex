\begin{frame}{Важные аббревиатуры внутри командной строки}
  \begin{itemize}
    \item Для директорий
      \begin{itemize}
        \item {\tt $\sim$} Домашняя директория
        \item {\tt $\sim$<username>} Домашняя директория пользователя
        \item {\tt ..} Родительская директория
        \item {\tt .} Текущая директория
      \end{itemize}
      \pause  
    \item Wildcards
      \begin{itemize}
        \item {\tt *} Любой набор символов {\tt file*txt : file1.txt filefilefiletxt}
        \item {\tt $[$<список>$]$ } символ из заданного набора
        \item {\tt ?} любой один символ
      \end{itemize}

  \end{itemize}
\end{frame}


\begin{frame}{Переменные окружения}
  \begin{itemize}
    \item {\tt HOME}
    \item {\tt PWD}
    \item {\tt LANG}
    \item {\tt LD\_LIBRARY\_PATH}
    \item {\tt SHELL}
    \item {\tt TERM}
    \item {\tt DISPLAY}
  \end{itemize}

  Контроль

  \begin{itemize}
    \item {\tt export VAR=value}
    \item declare -x
    \item echo \$VAR получить содержимое переменной
  \end{itemize}

  Переменные окружения наследуются при создании нового процесса
\end{frame}



