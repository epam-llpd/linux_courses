\begin{frame}[fragile]{Оболочка операционной системы}

     \begin{block}{Оболочка операционной системы}
     (от англ. shell «оболочка») \alert{интерпретатор команд} операционной системы, обеспечивающий интерфейс для взаимодействия пользователя с функциями системы.  
     \end{block}
     Что такое Unix shell?
     \begin{itemize}
        \item Обычная программа, запускающаяся после входа в систему
        \item Интерактивный командный интерпретатор
        \item Платформа интеграции (для утилит)
        \item Язык программирования
     \end{itemize}

	\pause
	\vspace{0.5in}
	Пример shell из  Windows-world --  cmd.exe, PowerShell
	\vspace{0.5in}
	Минимальный дистрибутив Linux -- ядро + shell 

\end{frame}
\note{   
     Как вы помните задачи ядра - управление процессами и потоками, управление
     памятью, Управление файлами, Разграничение доступа, мониторинг и
     конфигурация. Shell обеспечивает пользователю  интерфейс доступа к этим функциям ядра.    
     Как Интерактивный командный интерпретатор shell предосталяет средства:
     редактора командной строки, часто со средствами упрощающими ввод команд
     такими как  горячие клавишами, автодополнение, история команд и поиск по
     истории, сокращения команд (aliases), сокращения путей к файлам c помощью
     спецсимволов. 
     Как Платформа интеграции предоставляет средства пренаправления
     ввода/вывода, каналы (pipes), и возможность создавать скрипты.
     Одновременно является языком программирования, но эту часть будем
     рассматривать позже. 
}

\begin{frame}[fragile]{Задание. Виды оболочек.}
Получить список установленных оболочек.
\begin{lstlisting}[language=bash]
cat /etc/shells
ls -l <filename> # для каждого элемента /etc/shells
readlink -e <filename> 
\end{lstlisting}
Запустить любую из установленных оболочек. 

\begin{lstlisting}[language=bash]
/usr/bin/ksh 
/usr/bin/zsh 
/usr/bin/fish
\end{lstlisting}

выход из оболoчки  Ctrl-D, либо exit
\end{frame}


\begin{frame}[fragile]{Основные типы shell в Unix}
  \begin{itemize}
    \item Bourne shell совместимые (POSIX-совместимые)
      \begin{itemize}
        \item \textbf{sh} исходная bourne shell (Steve Bourne, 1978)
        \item \textbf{ksh} Korn shell (David Korn, 1983)
        \item \textbf{ash} $[$BSD$]$ Almquist shell (Kenneth Almquist,1989)  
        \item \textbf{bash} $[$GPL$]$ Bourne-again shell (Brian Fox, 1989)
        \item \textbf{zsh} $[$BSD$]$ Z shell (Paul Falstad,1990)
        \item \textbf{/bin/sh} Указывает на POSIX-совместимую shell
      \end{itemize}
  \item C shell совместимые
      \begin{itemize}
        \item \textbf{csh}  Исходная С shell (Bill Joy, 1978)
        \item \textbf{tcsh} $[$BSD$]$ TENEX C shell (Ken Greer, 1981)
       \end{itemize}
  \end{itemize}
\end{frame}

\note { 
Существуют  Рассмотрим наиболее распрастраненные реализации оболочек: 
Разделяются на две группы POSIX-совместимые и несовместимые. 
sh — оригинальный шелл Борна;
bash (bourne again shell) (эмуляция совместимости POSIX[1]) расширенная Борном свободная (разработанная в рамках проекта GNU). Стандартная оболочка в Linux.

C shell — (несовместима с POSIX shell) оболочка, с синтаксисом на основе Си,
созданная Университетом Беркли в рамках проекта по реализации BSD Unix.
    csh (C-Shell) — оболочка из состава дистрибутива BSD, имеет Си-образный синтаксис и не является POSIX-совместимой. Впервые введены возможности управления заданиями и произведены другие улучшения.
    tcsh (csh) — реализация csh с интерактивными возможностями, не уступающими bash[1]. Удобна для интерактивной работы. Совместима с csh.
}

\begin{frame}[fragile]{Различия между оболочками}
  \begin{itemize}
    \item Интерактивные возможности,  автодополнение, подсветка синтаксиса, клавиатурные сокращения
    \item Встроенные команды 
    \item Совместимость c POSIX или между собой
    \item Поиск соответствий строк и имён файлов, подстановки, globbing
    \item Средства программирования
\end{itemize}
\end{frame}
/note {
Интерактивные возможности - клавиатурные сокращения, средства автодополнения, настройки по умолчанию
Платформы - на чем работают оболочки. bash - кросплотформенная, zsh - нет. 
Программные возможности - различный синтаксис, bashизмы, функции определяются
по разному, средства программирования циклы, списки, массивы. Отсюда
совместимость между собой. 
Мы будет придерживаться синтаксиса оболочки bash, т.к. она по умолчанию
используется в дистрибутивах Linux.  Это позволит создать базу, с
котороый вы легко сможете перейти на подходящие вам варианты. 
}

\begin{frame}[fragile]{Задание. Виды команд в оболочке}
Выполним команду type для разных команд.
\begin{lstlisting}[language=bash]
type type cd help alias read
type dmesg rm
type if
type -a ls
type -a echo pwd test
\end{lstlisting}
\end{frame}

\begin{frame}[fragile]{Запуск программы из командной строки}
  \begin{itemize}
    \item 
	Находим приглашение командной строки
	\$, \#, user@host:~\$
    \item
	Вводим имя команды, аргументы и запускаем на выполнение нажатием <Enter>
   \end{itemize}

	Что такое команды?
  \begin{itemize}
    \item исполняемая программа (бинарный файл, скрипт)
    \item встроенные в оболочку команды (shell built-ins)
    \item функция оболочки
    \item сокращение команды (an alias) 
  \end{itemize}
\end{frame}
\note {
Система ожидает ввода команды, показывая приглашение командной строки.
Пользователь вводит команды опции аргументы и нажимает <Enter>. 
Опции - модификаторы поведения программы.  
Аргументы - то над чем производятся действия.
}
