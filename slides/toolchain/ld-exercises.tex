\begin{frame}[fragile]
	\frametitle{Задача}

	\begin{enumerate}
		\item Создать исходный файл({\tt libtesta.c}), реализующий функцию {\tt void test(void)}
		\item Создать исходный файл программы ({\tt main.c}), которая будет использовать функцию {\tt test()}
		\item Создать исполняемый бинарный файл
		\item Создать статическую библиотеку ({\tt libtesta.a})
			и статически связать ее c исполняемым бинарным файлом
		\item Создать динамическую библиотеку ({\tt libtesta.so.0.1.0})
			и связать ее c исполняемым бинарным файлом
		\item Создать исходный файл({\tt libtestb.c}), также реализующий функцию {\tt void test(void)},
			и создать динамическую библиотеку ({\tt libtestb.so.0.1.0})
		\item Изучить влияние переменных окружения и опций линковки
	\end{enumerate}
\end{frame}

\begin{frame}[fragile]
	\frametitle{Упражнение: исполняемый бинарный файл}

\begin{verbatim}
gcc -c -o libtesta.o libtesta.c
gcc -c -o main.o main.c
gcc -o main-classic main.o libtesta.o
\end{verbatim}
\end{frame}


\begin{frame}[fragile]
	\frametitle{Упражнение: статическая библиотека}

	\begin{enumerate}
		\item Создать статическую библиотеку {\tt libtesta.a} с помощью {\tt ar}

\begin{verbatim}
gcc -c -o libtesta.o libtesta.c
ar rcs libtesta.a libtesta.o
\end{verbatim}
			
		\item Создать статическую программу программу, которая будет использовать 
			библиотечную функцию {\tt test()} из статической библиотеки: 

\begin{verbatim}
gcc -c -o main.o main.c
gcc -o main-classic-ar main.o -L. -ltesta
\end{verbatim}

		\item Создать статически исполняемый файл: 

\begin{verbatim}
gcc -o main-static main.o -L. -ltesta -static
\end{verbatim}

	\end{enumerate}
\end{frame}


\begin{frame}[fragile]
	\frametitle{Упражнение: динамическая библиотека}

	\begin{enumerate}
		\item Скомпилировать объектный файл {\tt libtesta.o} с опцией {\tt -fPIC}
		
\begin{verbatim}
gcc -c -fPIC libtesta.c -o libtesta.o
gcc -shared -Wl,-soname,libtesta.so.0 -o libtesta.so.0.0 libtesta.o
ln -s libtesta.so.0.0 libtesta.so
\end{verbatim}
		\item Создать динамически исполняемый файл с использованием динамической библиотеки\footnote{Hint: link name}
\begin{verbatim}
gcc -c -o main.o main.c
gcc -o main main.o -L. -ltesta
\end{verbatim}

	\end{enumerate}
\end{frame}


\begin{frame}[fragile]
	\frametitle{Упражнение: запуск с динамической библиотекой}

	\begin{enumerate}
		\item Переопределить переменную {\tt LD\_LIBRARY\_PATH}\footnote{Hint: soname}:\\
			{\tt LD\_LIBRARY\_PATH=\$PWD ./main}
		\item Переопределить переменную {\tt LD\_PRELOAD}:\\
			{\tt LD\_PRELOAD=\$PWD/libmylib.so.0.0 ./main}
		\item Добавить опцию {\tt -rpath} при линковке:\\
			{\tt gcc hello.o -L. -lalib -Wl,-rpath=\$PWD -o hello}
		\item Переместить библитеку в системную область ({\tt /lib64, /usr/lib64, /usr/local/lib64}
	\end{enumerate}
\end{frame}


%\begin{frame}
%  \frametitle{Упражнение}
%  \begin{itemize}
%    \item Оформить функцию {\tt hello(char * user) } которая приветствует пользователя в отдельный файл
%    \item Скомпилировать этот файл и создать из него разделяемую динамическую библиотеку {\tt (libhello.so)}
%    \item Создать программу вызывающую функцию {\tt hello()} и слинковать ее с динамической библиотекой
%  \end{itemize}
%\end{frame}


\begin{frame}[fragile]
	\frametitle{Упражнение: последовательность линковки библиотек}

	\begin{enumerate}
		\item Создать 2 библиотеки ({\tt libtesta} и {\tt libtesta})
			с одинаковой функцией {\tt void test(void)}, но разным функционалом\\
			(например вывести разные строки на экран)
		\item Создать программу ({\tt main.c}), которая будет использовать 
			библиотечную функцию {\tt test()}
		\item Линковать с библиотеками в различной последовательности и 
			посмотреть результат выполнения команды

\begin{verbatim}
gcc -L. -Wl,-rpath,. -Wl,--no-as-needed main.o -ltesta -ltestb -o main-A_B
gcc -L. -Wl,-rpath,. -Wl,--no-as-needed main.o -ltestb -ltesta -o main-B_A
\end{verbatim}
	\end{enumerate}
\end{frame}



